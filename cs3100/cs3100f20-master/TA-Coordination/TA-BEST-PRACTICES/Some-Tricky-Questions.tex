
% --------------------------------------------------------------
% This is all preamble stuff that you don't have to worry about.
% Head down to where it says "Start here"
% --------------------------------------------------------------

\documentclass[12pt]{article}

\usepackage{graphicx,url}
\usepackage{proof}
\usepackage{framed}
\usepackage{etaremune}

\usepackage[margin=1in]{geometry}
\usepackage{amsmath,amsthm,amssymb,amsfonts}
\usepackage{paralist}

% 1. To get version suitable for students to populate,
%    remove the contents of the \ignoreSoln{..body..}
%
% 2. To get a version suitable for generating PDF 
%    without solutions, remove the #1 below
%
% 3. To generate solutions, keep the #1 below
%
% 4. Assigned grader fills \ignoreSoln{..body..}
%    and also provides his/her feedback to student
%    and policy followed for point deduction
%    So design policy before grading begins.

\newcommand{\ignoreSoln}[1]{#1}   
%\newcommand{\ignoreModel}[1]{#1} 


\newcommand{\bigset}[2]{\big\{\;#1\;:\;#2\;\big\}}
\newcommand{\N}{\mathbb{N}}
\newcommand{\Z}{\mathbb{Z}}
\newcommand{\R}{\mathbb{R}}
\newcommand{\Np}{\mathbb{N^{+}}}

\newenvironment{theorem}[2][Theorem]{\begin{trivlist}
\item[\hskip \labelsep {\bfseries #1}\hskip \labelsep {\bfseries #2.}]}{\end{trivlist}}
\newenvironment{lemma}[2][Lemma]{\begin{trivlist}
\item[\hskip \labelsep {\bfseries #1}\hskip \labelsep {\bfseries #2.}]}{\end{trivlist}}
\newenvironment{exercise}[2][Exercise]{\begin{trivlist}
\item[\hskip \labelsep {\bfseries #1}\hskip \labelsep {\bfseries #2.}]}{\end{trivlist}}
\newenvironment{reflection}[2][Reflection]{\begin{trivlist}
\item[\hskip \labelsep {\bfseries #1}\hskip \labelsep {\bfseries #2.}]}{\end{trivlist}}
\newenvironment{proposition}[2][Proposition]{\begin{trivlist}
\item[\hskip \labelsep {\bfseries #1}\hskip \labelsep {\bfseries #2.}]}{\end{trivlist}}
\newenvironment{corollary}[2][Corollary]{\begin{trivlist}
\item[\hskip \labelsep {\bfseries #1}\hskip \labelsep {\bfseries #2.}]}{\end{trivlist}}

\DeclareMathSizes{14}{14}{14}{14}

\begin{document}

% --------------------------------------------------------------
%                         Start here
% --------------------------------------------------------------

%\renewcommand{\qedsymbol}{\filledbox}


\begin{center}
\begin{large}
  CS 3100, Fall 2020 -- Some Tricky Questions \\
  \ \\
  \ \\  
      {  {\Large\bf NAME: } \hfill {\Large\bf UNID: }\hspace{4cm} }
      \ \\
  \ \\      
\end{large}


\end{center}
\date{}

\subsubsection*{Tricky Concepts and Practical Issues}

\begin{compactitem}
\item Some don't know Python; watch out!
\item Some come from the Windows world; they are troubled
  by Unix-style paths (forward-slash, etc). Largely avoided by
  Colab.
\item General weakness about strings. Things like
  empty set ($\emptyset$),
  empty string ($\varepsilon$),
  empty set containing empty string ($\{\varepsilon\}$),
  and set containing empty set  ($\{\emptyset\}$)
  are sure to confuse many.
  %
  While there is no standard recitation section,
  {\bf why not dedicate TA office hours early in the semester
    when you have no grading work to going over these?}
  %
  I'll prepare helpful recitation notes.
  
\item Concatenation versus intersection is confused.
\item Finite versus infinite is confused. The notions of
  finite/infinite sets of finite/infinite-strings (all combinations
  are worth thinking of) will confuse them.
\item In particular, the Kleene-star
  operation forms {\bf infinite sets of finite strings} 
\item The fact that a set can be non-regular but a superset
  of that set can be regular

\end{compactitem}

\subsubsection*{Tricky Questions}

\noindent Please try these problems ahead of time.
          {\bf IMPORTANT: Ask me for help if you get stuck.} {\em Please do not delay asking
          for help}.
          I'm OK if you don't know parts of this question; I'll help you. 

\begin{compactenum}

\item What are the differences, if any,
  between the language
  $L_1= (\{0\}\cup\{1\})^*$
  and the regular expression
  $(0+1)^*$? Learn to explain to me, then the student.

\item  How do the above relate to the language
  $L_2= (\{0\}^* \{1\}^*)^*$
  and the regular expression
  {\tt (0* 1*)*}? Learn to explain to me, then the student.

\item List 10 strings in numeric order from $L_1$ and $L_2$.

\item Argue that $L_1 = L_2$ --- first informally, then via a rigorous proof.
  Your proof can take advantage of the fact that $\{0,1\}\subseteq L(0*\; 1^*)$
  where $L$ stands for ``Language.'' (You can come back to this rigorous proof later.)

  
\item  Learn how to explain this NFA to a student. Walk through each line in the
  NFA's construction.
%--  
\begin{verbatim}
testnfa = md2mc('''
NFA
I  : 0 -> I
I  : 1 -> I, S0, A
A  : 0 -> I, A
A  : '' -> I
S0 : 0  -> F
''')
dotObj_nfa(testnfa, FuseEdges=True)
 \end{verbatim}

\item Now, learn how to explain the drawing of {\tt testnfa} generated above to the student.

\item Learn how to explain to the student how this construction works. That is, how exactly is
{\tt testnfa} being turned into the DFA called {\tt d}? Present
the subset construction
algorithm to me (then later, the student).
%--
\begin{verbatim}

d = nfa2dfa(testnfa)

dotObj_dfa(d)

 \end{verbatim}


\item How is this DFA getting minimized? Learn the full Dynamic Programming algorithm and
explain it to me.
%
\begin{verbatim}

md = min_dfa(d)

dotObj_dfa(md)

 \end{verbatim}

\item Show that if a language is regular, its complement is regular.
Define all the terms involved.

\item Building DFA for bit-strings that arrive at you MSB-first
  where the value of the number seen so far must be equal to $0$
  modulo $3$. Study this construction from the book and make sure you follow it,
  and then  learn how to explain to the student.

\item Building DFA for bit-strings that arrive at you LSB-first
  where the value of the number seen so far must be equal to $0$
  modulo $3$.  Study this construction from the book and make sure you follow it.
  The book only sketches a part; please finish it. Ask my help if you wish.

\item Showing that the concatenation of two context-free
  languages is context-free.

\item ({\bf Hard, till you see how easy it is}) Showing that
  the concatenation of two non-regular languages can be regular.
 %
 Hint: How about if you took the language $L = \{0^i \;:\; i\; {\rm is}\; {\rm prime}\}$.
 First, show $L$ is not regular. Then consider $\overline{L}$ ($L$'s complement). Can it be regular?
 Then consider $L\overline{L}$: what strings are in it? Finish your proof.
  
\item Show that the language
  \[ \{ a^i b^j c^k \;:\; i,j,k\ge 0,\; {\rm and}\; (i==3?(j==k):true) \}\]
  cannot be shown to be non-regular via the textbook Pumping Lemma
  that easily. However
  \begin{compactitem}
  \item If you intersected it with $b^*\; c^*$, the proof becomes
    easy.
  \item If you reversed the language, the proof becomes easy.
  \end{compactitem}
  %
  Read the relevant parts from the book.

\item Developing an NFA for the language of strings over $\{0,1\}$
  where the $n$-th last bit is a $1$ (for various $n$). Do this in Jove.

\item Show that if a language is context-free, its complement is not
  necessarily context-free. You may argue by taking an example CFL
  and show that its complement is not a CFL (actually prove it is not
  a CFL). You'll need to use the CFL PL.

\item Build a DTM for the language of strings of the form
  $w\#x$ where $w$ is a substring of $x$.

\item Next, build an NDTM for the above language. (Doing it for an NDTM
  is easier.)

\item How can you obtain a DFA $D$ corresponding
to the middle-third of all strings accepted by another DFA $D^{'}$?
\end{compactenum}

\end{document}
