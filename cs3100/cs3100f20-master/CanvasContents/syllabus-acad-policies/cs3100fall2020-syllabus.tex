\documentclass[12pt]{article}

\usepackage{url}
\usepackage{paralist}
\usepackage{times}
\usepackage{hyperref}
\usepackage[margin=1in]{geometry}

%\newcommand{\para}[1]{{\vspace{4pt}\noindent\bf{#1.}\hspace{4pt}}}
\newcommand{\para}[1]{{\vspace{4pt}\noindent\bf{#1:}}}
\newcommand{\denselist}{\vspace{-5pt} \itemsep -2pt\parsep=-1pt\partopsep -2pt}


\begin{document}


\begin{center}
{\Large CS 3100: Models of Computation  \\ Course Syllabus and Administrative Details\\
\vspace{2mm}
Fall 2020}\footnote{Syllabus adapted from Spring CS 2100's}
\end{center}

\begin{sl}
\subsubsection*{Updates of 8/26/20}
Please join Piazza for conducting Live Q\&A -- the recommended
way to ask questions and
discuss with classmates during my lecture.
%
I'll monitor the questions during class (the TAs will, mainly).
%
I'll go thru these posts after class
and highlight some for follow-up discussions as necessary.
%
I'll limit Zoom messages to be private posts to me, in response to
periodic questions I'll be displaying on my slides. More on
Section~\ref{sec:messaging}.
\end{sl}

\section{We care for your safety!}

This class is fully online, thus permitting
physical isolation during these trying times
caused by COVID-19.
%
Unfortunately, it also robs us of opportunities
for inter-personal interactions, and so please stay
in sync with as many online lectures as you can through
enthusiastic participation.


Kindly guard your physical and mental health, and we wish you
and your near and dear all the strength to cope
with the added stress,
seeking help from
\url{https://coronavirus.utah.edu} and
\url{counselingcenter.utah.edu}
as needed.
%
The Center for Disability and Access now has an online portal
\url{https://disability.utah.edu/online.php}
through which you can seek accommodations.
%
See the instructor's contact info below.



\section{Important Information}

\para{Class Website} Canvas (available through CIS)\\
\noindent\textbf{All course materials are available on Canvas or links pointed to from Canvas.}
The main URL used will be
\url{https://github.com/ganeshutah/Jove/tree/master/For_CS3100_Fall2020}
and hereafter called \href{https://github.com/ganeshutah/Jove/tree/master/For_CS3100_Fall2020}{Jove Website for Class}.
%
In particular, given that this class has Jupyter-based software for Automata
(called ``Jove''), and much of the learning is accomplished by interacting
with this software, we will
provide you detailed lesson plans via the above Github site.
%
We will also be setting and delivering all assignments through this site.
%
There will be a brief PDF that summarizes the assignment in the assignment
submission portal.
%
{\em For details on Jove, Colab, and Python, see Section~\ref{sec:detailed-info}.}



\para{Instructor} Ganesh Gopalakrishnan \\
\indent Physical Office: None (The class is virtual) \\
\indent Contact:  Via {\bf Canvas} Email
(use {\tt ganesh@cs.utah.edu} if you cannot get onto Canvas for some reason).

\para{Class Meeting Time} 
\begin{compactitem}
\item Tuesdays, Thursdays, 10:45 AM - 12:05 PM
\item[] Class assembles on Zoom. Zoom Links are published on Canvas.
  {\em Please attend as many lectures as you possibly can.}
  We will try and keep Zoom cloud recordings on Canvas
  (to be FERPA-compliant).
  They will be deleted after 30 days.
\end{compactitem}

\para{TA And Instructor Office Hours:}
On Zoom, and the timings + Zoom links are maintained
at \url{bit.ly/cs3100f20_office_hours}


\para{Textbook} (required) Authored by the instructor,
and a PDF is provided with permission of the publisher free
of charge on Canvas. The PDF is for use only during Fall 2020.


\para{Important Dates:\/} Kindly see \href{https://github.com/ganeshutah/Jove/tree/master/For_CS3100_Fall2020}{Jove Website for Class}


\para{Final Course Grade (IMPORTANT)}
\begin{compactitem}
% \item {\bf Participation:} 5 points. 
% \item[]  I am requiring one submission per lecture.
%   A total of 20 submissions over the semester earns you full points.
%   %
%   {\em There is no fixed submission}; I am seeking a short
%   submission of an {\tt .ipynb} notebook capturing your understanding
%   of a lecture.
%   %
%   Kindly visit \verb|03_Advanced_DFA/Another_Illustration.ipynb|
%   and that is roughly the ``short'' size I am looking for!
%   %
%   What items may you submit? Here are your choices:
%   \begin{compactitem}
%   \item A Jove notebook
%     that contains a simple {\em variant} of an example that we discussed
%     in class.
%     Fully run this on Colab, download the {\tt .ipynb} file and submit.
%     
%   \item Some of you may want to say something about a concept you learned.
%     Even these must be typed in markdown notation into a notebook.
%     
%   \end{compactitem}
% \item[] I will browse all your submissions, but 
%   only provide feedback selectively.

\item One ungraded survey issued on Canvas roughly weekly; 15 are {\em required};
  may qualify for bonus points.
\item 14 Quizzes (top 10 count; 2 points each), for 20 points.
\item 12 Assignments (top 10 count, 4 points each), for 40 points.
\item[]
\item[] {\bf HOW TO SUBMIT ASSIGNMENTS}
  \begin{compactitem}
  \item Put all the items you are asked to submit in a folder whose
    name is also specified clearly in the assignment submission PDF.
  \item Zip up that folder. Instructions to zip a folder
    are found on the web; here is
    one example:
    \href{https://www.hellotech.com/guide/for/how-to-zip-a-file-mac-windows-pc}{HERE}.
  \item Turn in that single Zipped folder.
  \item Our assignment submission portal will accept only ONE Zip file!
  \end{compactitem}
\item[]  
\item MT-1 for 12 points; 9/22/20 (exam is take-home)
\item MT-2 for 12 points; 10/22/20 (exam is take-home)
\item Finals for 16 points (Exam Thursday 12/10/20, 10:30am - 12:30pm on Canvas)
\end{compactitem}
\noindent Website
\href{https://github.com/ganeshutah/Jove/tree/master/For_CS3100_Fall2020}{Jove Website for Class}
has the details of all assignments and exams. Kindly make a calendar out of them.

\para{Prerequisites} CS 2100 or permission of instructor.



\section{Course Information}

CS 3100 provides 
you a rigorous introduction to various
models of computation.
The presentation will be hands-on using
Jupyter notebooks (collectively
called ``Jove'') authored by the author,
with help in strategic aspects by his students
(e.g. Jove's animation by Paul Carlson; many
Jove Automata defined by Ian Briggs; etc. A
fuller list is in his book's preface.)
%
Both Jupyter and Jove will be used interchangeably
and largely means the same; however,
the latter connotes the particular packaging
of learning material offered in this course
(e.g., specific conventions and path-settings
needed for the course software to work).


\para{Fair Warning} The pacing in this class is brisk.
Students should be aware that not all of the topics they
need to know will be covered during lectures.
Students should spend a considerable amount of time reading,
watching videos, studying, and solving problems outside of lecture.
%
Also this is a {\em flipped class} in that a student who comes
to class without having done the pre-assigned work is likely to struggle
and may not follow the couse material.

\section{Course Materials}

\para{Website} The class website
is always under development, with updates to the class schedule,
course notes, homework specifications, and more, occurring regularly.
It is critical that students become familiar with the class website
right away and plan to visit it several times a week, at a minimum.

The URLs are as follows:
\begin{compactitem}
\item  Website for External Reference: \url{bit.ly/cs3100fall2020} (contains this file's PDF
   and nothing more!)
\item Official Website with Weekly Lessons and Dates: What lists when you visit
  the
\href{https://github.com/ganeshutah/Jove/tree/master/For_CS3100_Fall2020}{Jove Website for Class}
\end{compactitem}

\para{Personal computers} The course needs the use of a computer
that can access the web. Many exercises can be completed
based on Google's Colab without any software installations.
The student is better off installing the Jove software
on their own laptop, as it runs faster and does not need web
connectivity (except for things like watching embedded Youtube videos).

Installation information is kept at
\href{https://github.com/ganeshutah/Jove/tree/master/For_The_Public/Classic}
{Jupyter\_Notebook\_Installation.pdf}


\section{Required Work Each Week}
\label{sec:messaging}

\begin{compactitem}
\item {\em Please participate in each lecture} by posting a
  on the Piazza Live Q\&A channel.
  %
  I'll monitor the questions during class.
  %
  Please confine Zoom messages to {\em private} replies to questions
  that I'll ask you
  during the lecture.

\item I will invite participation from students by running a Colab session of their own,
  and asking them to try a few things. If it works, you can give a {\bf thumbs-up}
  to acknowledge me.
  
\item The student must monitor the items listed in the
  \href{https://github.com/ganeshutah/Jove/tree/master/For_CS3100_Fall2020}{Jove Website for Class}, as things may change   
 frequently.

\item They must do the quizzes that pertain to required reading +
  video-watching and Jove-notebook practice {\bf before} a pair of
  lectures. Unless you run these Jove notebooks and do the reading,
  you will not be able to do the quizzes successfully.

\item The lectures will further drive the point home, leaving you with
  an assignment to solve and submit each week.

\item This schedule will be punctuated by two midterm exams whose coverage
  is clearly mentioned. The final exam will cover the last third of the
  course (roughly) but will also have some questions from prior
  portions (to check your higher level assimilation of ideas).
\end{compactitem}

\noindent Suggested steps for approaching CS 3100 homework assignments:

\begin{enumerate}\denselist
\item Read the relevant sections of the textbook in a timely way.
\item Be on top of the Jove notebook practice.
  The URL
  \url{https://github.com/ganeshutah/Jove/tree/master/For_The_Public}
  (the ``Jove Website for Others'')
  contains additional examples that you can take advantage of.
\item Each assignment will be accompanied by detailed instructions
  on what is expected. Ask for timely clarifications.
\item If you are still struggling after step 4, make use of the
  instructor's office hours and/or the TA help hours
  (see the class website for schedules).
\end{enumerate}

Homework assignments are to be done independently.
It is acceptable for students to discuss how to solve problems
with classmates, but copying solutions is considered academic misconduct.
It is the student’s responsibility to ensure the successful and timely
submission of each assignment via 
\textbf{Canvas}.
%
Corrupted or missing files are not grounds for extensions —
double-check your submissions and save a digital copy of all
of your work in your CADE account.\\

\para{Regrades} 
Students who wish to appeal a score on a homework assignment or an in-class quiz must do so within \textit{one week} of receiving the score via
Gradescope or Canvas.\\

\para{Letter grades} 
The following table is used to associate numerical scores with the corresponding letter grade. \textbf{Note the lack of rounding}. 
Let $x$ denote a student's numerical score. 

\noindent
{\bf A} $100 \geq x \geq 93$ {\bf A-} $93 > x \geq 90$ \\	
{\bf B+} $90 > x \geq 87$ {\bf B} $87 > x \geq 83$ {\bf B-} $83 > x \geq 80$ \\
{\bf C+} $80 > x \geq 77$ {\bf C} $77 > x \geq 73$ {\bf C-} $73> x \geq 70$ \\
{\bf D+} $70 > x \geq 67$ {\bf D} $67 > x \geq 63$ {\bf D-} $63 > x \geq 60$ \\
{\bf E} $60 > x$\\

\section{Getting Help}

To get help understanding course material, students may see the Teaching Assistant(s) during TA Help Hours, see the instructor during Office Hours, post a question to the Q\&A forums on Piazza (https://piazza.com), or contact the course staff directly (also via Piazza).
%
{\bf Canvas will be used only for teacher/student/TA emails. Canvas
  will not be used for any discussions.}


We will not be monitoring Piazza during the lectures, nor do we recommend
that you be distracted by it!


\section{Policies and Guidelines}

\begin{compactitem}\denselist
\item CS 3100 Academic Misconduct Policy: See on Canvas.

\item School of Computing Policies and Guidelines: \\
  \small{\url{https://handbook.cs.utah.edu/2019-2020/Academics/policies.php}}
  
\item College of Engineering Guidelines:\\
  \small{\url{www.coe.utah.edu/students/academic-affairs/academics/semester-guidelines}}
  
\item UofU Student code:\\
  \small{\url{www.regulations.utah.edu/academics/guides/students/studentRights.html}}
  
\end{compactitem}
Students should read and understand each of these documents, asking questions as needed.




\section{Detailed Info on Colab, Jove, and Python}
\label{sec:detailed-info}


A file that ends with extension {\tt .ipynb} is called {\em a Jupyter notebook}.
Sometimes we will also call it the {\em Jove notebook} or
sometimes merely ``{\em a notebook file}.''
  %
Here are the steps to execute a  notebook file.
%--
\begin{compactitem}
  
\item Click on 
\href{https://github.com/ganeshutah/Jove/tree/master/For_CS3100_Fall2020}{Jove Website for Class}
  and browse it via a browser. The contents of the README will be
  listed after the directories (folders). That tells you what you'll be doing each week.

\item Next, (if you don't already have it), obtain a Colab extension to your browser:
  \begin{compactitem}
  \item Here is the \href{https://chrome.google.com/webstore/detail/open-in-colab/iogfkhleblhcpcekbiedikdehleodpjo?hl=en}{Chrome Colab Extension}
  \item Here is the \href{https://addons.mozilla.org/en-US/firefox/addon/open-in-colab/}{Firefox Colab Extension}
  \end{compactitem}

\item When you are visiting a {\em notebook file} ({\tt .ipynb file})
  via {\em Github}'s file listing, click on the Colab extension. This
  will load the file contents onto Colab, and then you can execute that
  file under Colab.
  
\item {\em Please become familiar with our textbook!} It is your most important resource
  in this class, and is required regular reading!
  %
  Our book summarizes a few aspects of Python, and also points to a good tutorial.
  It also summarizes virtually all the Jove functions in its Appendices.

  
\item  To know more about Colab, kindly visit the
  \href{https://colab.research.google.com/notebooks/intro.ipynb#}{Colab Tutorial}.
  You will be given working knowledge of Colab during our lectures.

\end{compactitem}


\subsection{Basic Python/Colab Experience You Need}
You'll need only a few facilities, and they are fully described here:

\begin{compactenum}
\item Setting up a browser extension: see above.
\item Click the browser extension as said above: you are put in a Python environment
  in Colab.
\item Then you will need the facilities to execute cells that
  are clearly mentioned in this video
  by Corey Schafer at \url{https://youtu.be/HW29067qVWk}.
  %
  You can pretty much go to the {\em 4:00 min mark} and directly
  learn how to use the keyboard shortcuts to run the cells.\footnote{You may
    install Jupyter on your laptop following instructions
    kept at {\tt Jove/For\_The\_Public/Classic/Jupyter\_Notebook\_Installation.pdf.}}
\item You'll occasionally need to write markdown commands. Learn that from
\item[]    
{\footnotesize \verb|For_CS3100_Fall2020/01_Computability_Languages/1a_Jupyter_Basics.ipynb|}

\item Finally, for all assignments, you must
  click \verb|File->Download .ipynb|, and save the notebook.
\end{compactenum}

\end{document}
%--end

