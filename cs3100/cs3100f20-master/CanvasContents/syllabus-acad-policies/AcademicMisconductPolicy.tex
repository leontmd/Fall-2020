\documentclass[11pt]{article}
\usepackage{fullpage,url}
\usepackage[colorlinks=true,urlcolor=blue,bookmarks=false,pdffitwindow=false]{hyperref}

\usepackage[letterpaper,top=1in,bottom=1in,left=1in,right=1in,nohead]{geometry}

\setlength{\parindent}{0in}
\setlength{\parskip}{6pt}

\begin{document}

\begin{center}
\huge{{\bf CS 3100: Models of Computation}}
\vspace{0.25\baselineskip}

\large{{\bf Academic Misconduct Policy}\\
Fall 2020}
\end{center}
\vspace{1.1\baselineskip} 

{\bf \large Definition of academic
  misconduct}\footnote{Adapted from a similar document used in CS 2100.} 

As defined in the University Code of Student Rights and
Responsibilities, academic misconduct includes, but is not limited to,
cheating, misrepresenting one's work, inappropriately collaborating,
plagiarism, and fabrication or falsification of information. It also
includes facilitating academic misconduct by intentionally helping or
attempting to help another to commit an act of academic
misconduct.\footnote{http://www.regulations.utah.edu/academics/guides/students/studentRights.html}
A primary example of academic misconduct would be submitting as one's
own, work that is copied from another student or an outside source.
In CS 3100, students are encouraged to discuss high-level strategies
for solving assignments\footnote{The term {\em assignment} as used throughout this document refers to any homework assignment, quiz, or final exam in CS 3100.}
with fellow classmates, but each student is
responsible for writing his/her own answer.

{\bf Academic misconduct is not:}
\begin{itemize}
\vspace{-.15in}
\item Communicating with classmates about assignments orally, in a spoken language like English.
\item Discussing the course material with others, so that they and you may understand it better.
\item Helping a classmate through discussing problems worked in class and not on the homework. 
\item Using the web and other resources for instruction beyond lecture/discussion, for references, and for solutions to technical difficulties, but not for outright solutions to assignments.
\item Working with a tutor, provided the tutor does not do the assignments for you.
\end{itemize}

{\bf Academic misconduct is:}
\begin{itemize}\vspace{-.15in}
\item Asking a classmate to see his/her solution to an assignment before submitting your own.
\item Viewing a classmate's solution to an assignment and basing your own solution on it.
\item Giving or showing to a classmate a solution to an assignment when it is he/she, and not you, who is struggling to solve it.
\item Providing or making available solutions to assignments to individuals who might take this course in the future.
%\item Uploading assignment solutions to web-based repositories such as GitHub (https://github.com),
%Bitbucket (https://bitbucket.org), and others.
\item Posting questions about assignments to any forums other than the Discussions on the CS 3100 class website.
\item Posting answers to assignments anywhere.
\item Searching for or soliciting outright solutions to assignments online or elsewhere, including from students who took this course in the past.
\item Splitting an assignment's workload with another individual and submitting a combination of
his/her work and yours.
\item Looking at another individual's work during a quiz or the final exam.
\item Searching for, soliciting, or viewing quiz or final exam questions or answers prior to taking that test or final exam.
\item Trying to change an answer on a quiz after it is graded and submitting it for more credit.
\item Using resources during a test or final exam beyond those explicitly allowed in the instructions.
\item Paying or offering to pay an individual for work that you may submit as (part of) your own.
\item Intentionally submitting a corrupted file as a scheme to get more time to work on an assignment.
\end{itemize}

{\em A good rule of thumb.} If you discuss an assignment with others, you must leave the discussion with nothing written or typed. This is the best way to ensure that you construct your own solution, and that the work you submit honestly represents your understanding of the course material.

If you are unsure what qualifies as academic misconduct, you are always welcome to talk to the
CS 3100 instructors. {\em There is absolutely no penalty for asking about a particular action, even if it
is academic misconduct, so long as you seek clarification before acting.}

\vspace{.1in}
{\bf \Large Sanction for academic misconduct and appeals process
}

{\bf For academic misconduct in CS 3100, the sanction is to fail the course.}

Upon discovering the misconduct, the instructor will discuss the
infraction with the student within 20 days. Within the next 10 days
the instructor will give written notice to the student describing the
sanction and advising him/her of his/her right to appeal. The
instructor will also notify the {\bf Senior Vice President of Academic
Affairs} of the infraction and the sanction, forwarding a copy to the
{\bf Director of the School of Computing}. Finally, a letter {\em describing the
infraction will be placed in the student's permanent School of
Computing academic record}.  As described in the College of Engineering
Academic Appeals and Misconduct Policy\footnote{http://www.coe.utah.edu/appeals}, the student has the right to
appeal any academic action he/she feels is arbitrary and capricious.

\end{document}
