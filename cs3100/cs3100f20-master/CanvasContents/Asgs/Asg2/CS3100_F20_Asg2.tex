
% --------------------------------------------------------------
% This is all preamble stuff that you don't have to worry about.
% Head down to where it says "Start here"
% --------------------------------------------------------------

\documentclass[12pt]{article}

\usepackage{graphicx,url}
\usepackage{proof}
\usepackage{framed}
\usepackage{etaremune}

\usepackage[margin=1in]{geometry}
\usepackage{amsmath,amsthm,amssymb,amsfonts}
\usepackage{paralist}
\thispagestyle{empty}

% 1. To get version suitable for students to populate,
%    remove the contents of the \ignoreSoln{..body..}
%
% 2. To get a version suitable for generating PDF 
%    without solutions, remove the #1 below
%
% 3. To generate solutions, keep the #1 below
%
% 4. Assigned grader fills \ignoreSoln{..body..}
%    and also provides his/her feedback to student
%    and policy followed for point deduction
%    So design policy before grading begins.

\newcommand{\ignoreSoln}[1]{#1}   
%\newcommand{\ignoreModel}[1]{#1} 


\newcommand{\bigset}[2]{\big\{\;#1\;:\;#2\;\big\}}
\newcommand{\N}{\mathbb{N}}
\newcommand{\Z}{\mathbb{Z}}
\newcommand{\R}{\mathbb{R}}
\newcommand{\Np}{\mathbb{N^{+}}}

\newenvironment{theorem}[2][Theorem]{\begin{trivlist}
\item[\hskip \labelsep {\bfseries #1}\hskip \labelsep {\bfseries #2.}]}{\end{trivlist}}
\newenvironment{lemma}[2][Lemma]{\begin{trivlist}
\item[\hskip \labelsep {\bfseries #1}\hskip \labelsep {\bfseries #2.}]}{\end{trivlist}}
\newenvironment{exercise}[2][Exercise]{\begin{trivlist}
\item[\hskip \labelsep {\bfseries #1}\hskip \labelsep {\bfseries #2.}]}{\end{trivlist}}
\newenvironment{reflection}[2][Reflection]{\begin{trivlist}
\item[\hskip \labelsep {\bfseries #1}\hskip \labelsep {\bfseries #2.}]}{\end{trivlist}}
\newenvironment{proposition}[2][Proposition]{\begin{trivlist}
\item[\hskip \labelsep {\bfseries #1}\hskip \labelsep {\bfseries #2.}]}{\end{trivlist}}
\newenvironment{corollary}[2][Corollary]{\begin{trivlist}
\item[\hskip \labelsep {\bfseries #1}\hskip \labelsep {\bfseries #2.}]}{\end{trivlist}}

\DeclareMathSizes{14}{14}{14}{14}

\begin{document}

% --------------------------------------------------------------
%                         Start here
% --------------------------------------------------------------

%\renewcommand{\qedsymbol}{\filledbox}


\begin{center}
\begin{large}
  CS 3100, Spring 2020, Assignment 2 -- 100 pts total -- Due Dates on Syllabus
  \ \\
%  \ \\  
%      {  {\Large\bf NAME: } \hfill {\Large\bf UNID: }\hspace{4cm} }
%      \ \\
  \ \\      
\end{large}


\end{center}
\date{}


\noindent There are four graded problems.
Relevant files are in the directory \verb|ASSIGNMENT-2|, and they are

\verb|u1234567_asg2_Prob1_XL.ipynb| and

\verb|u1234567_asg2_Prob234.ipynb|

As with Assignment-1, Zip the folder and submit!

\noindent {\bf Example:} If your UNID is     \verb|u1234567|, then you'll be
making a folder named \verb|u1234567|, and putting the FULLY SOLVED
files

\verb|u1234567_asg2_Prob1_XL.ipynb| and

\verb|u1234567_asg2_Prob234.ipynb|

into that folder, Zip the folder, and submit.


%=================================================================
\begin{enumerate}
\item (25 points; Grader: Xinyi (Nicole))
  \noindent Assigned work:
 The Colab notebook to run and solve is
 \verb|u0000000_asg2_Prob1_XL.ipynb|.
 Here are the parts in this assignment (for details, see the notebook):
 \begin{compactitem}
 \item Python set comprehension definition for $L_6$
 \item Python set comprehension definition for $L_8$
 \item Code to check identity $L_6 \cup L_8 = \Sigma_3^{*}$
 \item Code to compute \verb|LangMissed|, the language missed
 \item Fully general mathematical description of the language missed, in Latex Markdown syntax.
 \item A necessary and sufficient condition for language concatenation
   to shrink in size compared to cartesian product.
 \end{compactitem}
 

  \item[] {\bf Finish the notebook for Xinyi (Nicole) and submit it!    }
\item[]
  
    
\item (75 points (three 25 point questions); Graders: Shashank, Archit, Leon)
  \noindent Assigned work:
 The Colab notebook to run and solve is
    \verb|u0000000_asg2_Prob234.ipynb|.
    This contains a sequence of three problems. Basically you'll be doing these:
    \begin{compactitem}
    \item Define a DFA that has an even number of 010s
    \item Define a DFA that ends in 010
    \item Intersect them
    \item Minimize them
    \item See if states were eliminated during minimization; if so, explain why
    \item Test the DFA using the \verb|accepts_dfa| function
    \item Learn the use of the Python \verb|filter| function
    \item If not enough strings accept, expand the testing range to include at least
      five accepted strings
    \end{compactitem}

  \item[] {\bf Finish the notebook as directed within, and submit it!    }
\item[]



\end{enumerate}
%=================================================================

\end{document}
