
% --------------------------------------------------------------
% This is all preamble stuff that you don't have to worry about.
% Head down to where it says "Start here"
% --------------------------------------------------------------

\documentclass[12pt]{article}

\usepackage{graphicx,url}
\usepackage{proof}
\usepackage{framed}
\usepackage{etaremune}

\usepackage[margin=1in]{geometry}
\usepackage{amsmath,amsthm,amssymb,amsfonts}
\usepackage{paralist}
\thispagestyle{empty}

% 1. To get version suitable for students to populate,
%    remove the contents of the \ignoreSoln{..body..}
%
% 2. To get a version suitable for generating PDF 
%    without solutions, remove the #1 below
%
% 3. To generate solutions, keep the #1 below
%
% 4. Assigned grader fills \ignoreSoln{..body..}
%    and also provides his/her feedback to student
%    and policy followed for point deduction
%    So design policy before grading begins.

\newcommand{\ignoreSoln}[1]{#1}   
%\newcommand{\ignoreModel}[1]{#1} 


\newcommand{\bigset}[2]{\big\{\;#1\;:\;#2\;\big\}}
\newcommand{\N}{\mathbb{N}}
\newcommand{\Z}{\mathbb{Z}}
\newcommand{\R}{\mathbb{R}}
\newcommand{\Np}{\mathbb{N^{+}}}

\newenvironment{theorem}[2][Theorem]{\begin{trivlist}
\item[\hskip \labelsep {\bfseries #1}\hskip \labelsep {\bfseries #2.}]}{\end{trivlist}}
\newenvironment{lemma}[2][Lemma]{\begin{trivlist}
\item[\hskip \labelsep {\bfseries #1}\hskip \labelsep {\bfseries #2.}]}{\end{trivlist}}
\newenvironment{exercise}[2][Exercise]{\begin{trivlist}
\item[\hskip \labelsep {\bfseries #1}\hskip \labelsep {\bfseries #2.}]}{\end{trivlist}}
\newenvironment{reflection}[2][Reflection]{\begin{trivlist}
\item[\hskip \labelsep {\bfseries #1}\hskip \labelsep {\bfseries #2.}]}{\end{trivlist}}
\newenvironment{proposition}[2][Proposition]{\begin{trivlist}
\item[\hskip \labelsep {\bfseries #1}\hskip \labelsep {\bfseries #2.}]}{\end{trivlist}}
\newenvironment{corollary}[2][Corollary]{\begin{trivlist}
\item[\hskip \labelsep {\bfseries #1}\hskip \labelsep {\bfseries #2.}]}{\end{trivlist}}

\DeclareMathSizes{14}{14}{14}{14}

\begin{document}

% --------------------------------------------------------------
%                         Start here
% --------------------------------------------------------------

%\renewcommand{\qedsymbol}{\filledbox}


\begin{center}
\begin{large}
  CS 3100, Spring 2020, Practice Problems 2 -- NOT Graded -- discuss
  \ \\
%  \ \\  
%      {  {\Large\bf NAME: } \hfill {\Large\bf UNID: }\hspace{4cm} }
%      \ \\
  \ \\      
\end{large}


\end{center}
\date{}


\noindent No submission is required; however be doing these problems
throughout the semester. The pertinent set of these problems will be reviewed
before each exam. {\bf The point values don't matter,} but I did not want to
destroy that info (may give you an idea of relative difficulties).

%=================================================================
\begin{enumerate}

\item Surprise! Selected problems at the end of Chapter-3 have been worked
  out for you! But first study them. After a good attempt on your
  part, seek their solutions at
  \url{bit.ly/Automata_Jove} under SolutionsToSelectedProblems.

 
\item[] {\bf Properties of Star}
  

%-----------------------------------------------------------------
\item \label{qn2}
  %~~~parts~~~
 \begin{enumerate}
 \item  \label{qn2a}
   Simplify the language subtraction
   \(  L_1 - L_2 \)
   where
   \[ L_1 = \{ wxw^R \;:\; w\in\Sigma^*\; {\rm and}\;
   x\in (\Sigma\cup \emptyset^*) \} \]
   and
   \[ L_2 = \{ wxw^R \;:\; w\in\Sigma^*\; {\rm and}\; x\in \Sigma\}\]
    

  
  \item  \label{qn2b}
    %
    Show that $L_1^* = \Sigma^*$ whereas $L_2^* \neq \Sigma^*$
 \end{enumerate}



\item[] {\bf Properties of DFA, Boolean Operations on DFA}
 
\item  \label{qn1}
  %~~~parts~~~
  \begin{enumerate}
  \item Consider any DFA with a single initial state $I$ and a single
    final state $F$
    %
    Let us swap these; that is, rename $I$ to $Fn$ to make it ``final''
    and rename $F$ into $In$ to make it ``initial.''
    %
    Now, turn all the edges of the DFA around; i.e. if there was a
    DFA edge from state $s_1$ to state $s_2$ via transition $a$,
    now erect a
    DFA edge from state $s_2$ to state $s_1$ via transition $a$.    
    %
    Argue that the result will no longer be a DFA.

  \item Suggest a simple way to turn a DFA for language $L$
    into a DFA for language $\overline{L}$.
    
  \item We defined the operations of union, intersection and complementation
    on DFA.
    %
    But we did not define the concatenation operation on DFA.
    That is, it seems difficult to imagine a simple ``graph surgery''
    on the graphs of DFA $D_1$ with language
    $L_1$
    and DFA $D_2$ with language $L_2$ to obtain a DFA
    for $L_1 L_2$.
    %
    What seems to be the difficulty?
    %
    List all the problems you notice by taking two example DFA, say
    a DFA $D_1$ for $\{aaa,aaaaa\}$ % 1,3,5
    and a DFA for
    $\{aa, aaaaaaa\}$ % 2,7
    over
    alphabet $\{a\}$.
  \end{enumerate}

\end{enumerate}


%=================================================================

\end{document}
