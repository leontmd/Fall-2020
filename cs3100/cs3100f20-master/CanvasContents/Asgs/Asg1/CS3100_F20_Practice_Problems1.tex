
% --------------------------------------------------------------
% This is all preamble stuff that you don't have to worry about.
% Head down to where it says "Start here"
% --------------------------------------------------------------

\documentclass[12pt]{article}

\usepackage{graphicx,url}
\usepackage{proof}
\usepackage{framed}
\usepackage{etaremune}

\usepackage[margin=1in]{geometry}
\usepackage{amsmath,amsthm,amssymb,amsfonts}
\usepackage{paralist}
\thispagestyle{empty}

% 1. To get version suitable for students to populate,
%    remove the contents of the \ignoreSoln{..body..}
%
% 2. To get a version suitable for generating PDF 
%    without solutions, remove the #1 below
%
% 3. To generate solutions, keep the #1 below
%
% 4. Assigned grader fills \ignoreSoln{..body..}
%    and also provides his/her feedback to student
%    and policy followed for point deduction
%    So design policy before grading begins.

\newcommand{\ignoreSoln}[1]{#1}   
%\newcommand{\ignoreModel}[1]{#1} 


\newcommand{\bigset}[2]{\big\{\;#1\;:\;#2\;\big\}}
\newcommand{\N}{\mathbb{N}}
\newcommand{\Z}{\mathbb{Z}}
\newcommand{\R}{\mathbb{R}}
\newcommand{\Np}{\mathbb{N^{+}}}

\newenvironment{theorem}[2][Theorem]{\begin{trivlist}
\item[\hskip \labelsep {\bfseries #1}\hskip \labelsep {\bfseries #2.}]}{\end{trivlist}}
\newenvironment{lemma}[2][Lemma]{\begin{trivlist}
\item[\hskip \labelsep {\bfseries #1}\hskip \labelsep {\bfseries #2.}]}{\end{trivlist}}
\newenvironment{exercise}[2][Exercise]{\begin{trivlist}
\item[\hskip \labelsep {\bfseries #1}\hskip \labelsep {\bfseries #2.}]}{\end{trivlist}}
\newenvironment{reflection}[2][Reflection]{\begin{trivlist}
\item[\hskip \labelsep {\bfseries #1}\hskip \labelsep {\bfseries #2.}]}{\end{trivlist}}
\newenvironment{proposition}[2][Proposition]{\begin{trivlist}
\item[\hskip \labelsep {\bfseries #1}\hskip \labelsep {\bfseries #2.}]}{\end{trivlist}}
\newenvironment{corollary}[2][Corollary]{\begin{trivlist}
\item[\hskip \labelsep {\bfseries #1}\hskip \labelsep {\bfseries #2.}]}{\end{trivlist}}

\DeclareMathSizes{14}{14}{14}{14}

\begin{document}

% --------------------------------------------------------------
%                         Start here
% --------------------------------------------------------------

%\renewcommand{\qedsymbol}{\filledbox}


\begin{center}
\begin{large}
  CS 3100, Spring 2020, Practice Problems 1 -- NOT Graded -- discuss
  \ \\
%  \ \\  
%      {  {\Large\bf NAME: } \hfill {\Large\bf UNID: }\hspace{4cm} }
%      \ \\
  \ \\      
\end{large}


\end{center}
\date{}


\noindent No submission is required; however be doing these problems
throughout the semester. The pertinent set of these problems will be reviewed
before each exam. {\bf The point values don't matter,} but I did not want to
destroy that info (may give you an idea of relative difficulties).

%=================================================================
\begin{enumerate}

\item Surprise! All the problems at the end of Chapter-2 have been worked
  out for you! But first study them. After a good attempt on your
  part, seek their solutions at
  \url{bit.ly/Automata_Jove} under SolutionsToSelectedProblems.

\item Surprise! A set of problems given out during previous midterms and final
  exams have been solved and kept online at the above URL, as well.
  
  
\item[] {\bf Language operations, counting.}
  

%-----------------------------------------------------------------
\item{\bf (22 points)} \label{qn2}
  %~~~parts~~~
 \begin{enumerate}
 \item {\bf 3 points:} \label{qn2a}
   (We need to properly understand the notion of powersets. This question
   helps you review that.)
   %
   Let $L=\{\varepsilon,a,c,ab\}$. How many elements are there in
   ${\cal P}(L)$ (the powerset of $L$)? Justify your answer in 1-2 clear sentences.
    

  
  \item {\bf 5 points:} \label{qn2b}
    %
    Define the $n$-fold cartesian product of a set $S$ to
    be $S\times S\ldots S$ ($n$ times).
    %
    Given this definition,
    is the cardinality of $\{a,\varepsilon\}^n$
    less than, equal to, or greater than the
    cardinality of the $n$-fold cartesian product
    of $\{a,\varepsilon\}$ with itself?
    %
    Carefully justify your answer in a few bulleted sentences.
 
   \item {\bf 3 points:} \label{qn2c}
    What is $L^R$ where the superscript
      $R$ denotes {\em reverse}? Write your answer out as a set.
 
    \item {\bf 11 points:} \label{qn2d}\footnote{Question contributed by TA Thanhson.}
      Show that $\{\varepsilon,a,b,c,\ldots,z\}^n$ has
      $(26^{n+1}-1)/25$ elements.
      %
      Hint: This formula comes from series summation of
      a geometric series: \\$(1+x+x^2+\ldots+x^n) = (x^{n+1}-1)/(x-1)$.
 \end{enumerate}



\item[] {\bf Properties of concatenation and Star.}
 
\item{\bf (14 points total):}  \label{qn1}
  %~~~parts~~~
  Consider the language $M=\{\varepsilon,a,b\}$. 
  \begin{compactenum}
  \item \label{M-question}
    {\bf (3 points)} \label{qn1a}
    Show that $MM \neq  M$.
    For two languages, $X$ and $Y$, $X\neq Y$ exactly when there is $s\in X$ that is
    not in $Y$, or vice-versa.
    Your answer must describe such an $s$ and justify in
    a sentence or two.

  \item 
    {\bf (4 points)} \label{qn1b}
    Argue that for all {\em non-empty}
    languages $N$ (over any alphabet) that don't contain $\varepsilon$, it is true that $N\neq NN$.

  \item 
    {\bf (7 points)} \label{qn1c}
    Argue that for a language $N$ over $\{a,b\}$ that contains $\varepsilon$,
    $a$, and $b$, we can have $N=NN$
    if and only if $N=\{a,b\}^*$.
    %
    {\em One approach:\/} Suppose not. That is, we are missing one string besides
    $\varepsilon$, $a$, and $b$. Assume $N=NN$ still holds. Derive a contradiction.



  \end{compactenum}



\end{enumerate}
%=================================================================

\end{document}
