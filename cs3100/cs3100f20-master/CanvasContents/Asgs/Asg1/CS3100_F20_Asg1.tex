
% --------------------------------------------------------------
% This is all preamble stuff that you don't have to worry about.
% Head down to where it says "Start here"
% --------------------------------------------------------------

\documentclass[12pt]{article}

\usepackage{graphicx,url}
\usepackage{proof}
\usepackage{framed}
\usepackage{etaremune}

\usepackage[margin=1in]{geometry}
\usepackage{amsmath,amsthm,amssymb,amsfonts}
\usepackage{paralist}
\thispagestyle{empty}

% 1. To get version suitable for students to populate,
%    remove the contents of the \ignoreSoln{..body..}
%
% 2. To get a version suitable for generating PDF 
%    without solutions, remove the #1 below
%
% 3. To generate solutions, keep the #1 below
%
% 4. Assigned grader fills \ignoreSoln{..body..}
%    and also provides his/her feedback to student
%    and policy followed for point deduction
%    So design policy before grading begins.

\newcommand{\ignoreSoln}[1]{#1}   
%\newcommand{\ignoreModel}[1]{#1} 


\newcommand{\bigset}[2]{\big\{\;#1\;:\;#2\;\big\}}
\newcommand{\N}{\mathbb{N}}
\newcommand{\Z}{\mathbb{Z}}
\newcommand{\R}{\mathbb{R}}
\newcommand{\Np}{\mathbb{N^{+}}}

\newenvironment{theorem}[2][Theorem]{\begin{trivlist}
\item[\hskip \labelsep {\bfseries #1}\hskip \labelsep {\bfseries #2.}]}{\end{trivlist}}
\newenvironment{lemma}[2][Lemma]{\begin{trivlist}
\item[\hskip \labelsep {\bfseries #1}\hskip \labelsep {\bfseries #2.}]}{\end{trivlist}}
\newenvironment{exercise}[2][Exercise]{\begin{trivlist}
\item[\hskip \labelsep {\bfseries #1}\hskip \labelsep {\bfseries #2.}]}{\end{trivlist}}
\newenvironment{reflection}[2][Reflection]{\begin{trivlist}
\item[\hskip \labelsep {\bfseries #1}\hskip \labelsep {\bfseries #2.}]}{\end{trivlist}}
\newenvironment{proposition}[2][Proposition]{\begin{trivlist}
\item[\hskip \labelsep {\bfseries #1}\hskip \labelsep {\bfseries #2.}]}{\end{trivlist}}
\newenvironment{corollary}[2][Corollary]{\begin{trivlist}
\item[\hskip \labelsep {\bfseries #1}\hskip \labelsep {\bfseries #2.}]}{\end{trivlist}}

\DeclareMathSizes{14}{14}{14}{14}

\begin{document}

% --------------------------------------------------------------
%                         Start here
% --------------------------------------------------------------

%\renewcommand{\qedsymbol}{\filledbox}


\begin{center}
\begin{large}
  CS 3100, Spring 2020, Assignment 1 -- 100 pts total -- Due Dates on Syllabus
  \ \\
%  \ \\  
%      {  {\Large\bf NAME: } \hfill {\Large\bf UNID: }\hspace{4cm} }
%      \ \\
  \ \\      
\end{large}


\end{center}
\date{}


\noindent There are four graded problems.
Relevant files are in the directory \verb|ASSIGNMENT-1|.
%
For each notebook containing your problem, replace the ``u0000000'' with your actual UNID,
then solve the contents of the files.
%
Then download the {\tt .ipynb} of the SOLVED notebook.
%
Then make a folder named u0000000, ZIP it, and submit it!


\noindent {\bf Example:} If your UNID is     \verb|u1234567|, then you'll be
making a folder named \verb|u1234567|, putting the
files

\verb|u1234567_asg1_Python_Refresher_AR.ipynb|,

\verb|u1234567_asg1_LangBasics_SV.ipynb|,

\verb|u1234567_asg1_langbasics_XL.ipynb|, and

\verb|u1234567_asg1_Demorgan_LT.ipynb|

into that folder, Zipping that folder, and turning that {\bf single zipped folder}.

%=================================================================
\begin{enumerate}
\item (25 points; Grader: Archit)
  \noindent Assigned work:
 The Colab notebook to run and solve is
    \verb|u0000000_asg1_Python_Refresher_AR.ipynb|.
 Read the description of the three languages defined in
    this Colab file. Express them in Python's set comprehension syntax
    and generate the specified finite languages.
    There is also one entry to be filled at the end -- {\bf the number of
      strings (and which strings) for which the condition $(i=2)$ is met.}

  \item[] {\bf Finish the notebook for Archit and submit it!    }
\item[]
  
    
\item (25 points; Grader: Shashank)
  \noindent Assigned work:
 The Colab notebook to run and solve is
    \verb|u0000000_asg1_LangBasics_SV.ipynb|.
    You have two tasks assigned inside these notebooks.

  \item[] {\bf Finish the notebook for Shashank and submit it!    }
\item[]

\item (25 points; grader: xinyi (nicole))
  \noindent assigned work:
 the colab notebook to run and solve is
    \verb|u0000000_asg1_langbasics_XL.ipynb|.
    You have two tasks assigned inside these notebooks.

  \item[] {\bf Finish the notebook for Xinyi (Nicole) and submit it!    }
\item[]

\item (25 points; grader: Leon)
  \noindent assigned work:
 the colab notebook to run and solve is
    \verb|u0000000_asg1_Demorgan_LT.ipynb|.
    Your task is defined in the notebook.

  \item[] {\bf Finish the notebook for Leon and submit it!    }        


\end{enumerate}
%=================================================================

\end{document}
